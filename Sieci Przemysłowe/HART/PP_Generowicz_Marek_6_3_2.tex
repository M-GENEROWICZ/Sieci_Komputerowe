\documentclass{article}
\usepackage{graphicx} % Required for inserting images
\usepackage[utf8]{inputenc}
\usepackage{polski}
\usepackage[dvipsnames]{xcolor}
\usepackage{indentfirst}
\usepackage{multicol}
\usepackage{geometry}
\usepackage{titlesec}
\usepackage[colorlinks=true, linkcolor=gray, urlcolor=blue, citecolor=green]{hyperref}
\usepackage{makecell}
\usepackage{float}
\usepackage[polish]{babel}
\usepackage[T1]{fontenc}
\usepackage[justification=centering]{caption}
\usepackage[utf8]{inputenc} 
\usepackage{subfig}


\usepackage{mwe} % for 'example-image'
\usepackage{newfloat}
\DeclareFloatingEnvironment{graph}
\addto\captionspolish{%
  \renewcommand{\graphname}{Wykres}%
  \renewcommand{\figurename}{Rysunek}%
}


\begin{document}

\begin{titlepage}
    \begin{center}
        \vspace*{1cm}
            
        \Huge
        \textbf{Sprawozdanie z laboratorium 6}
            
        \vspace{0.5cm}
        \LARGE
        HART (WAGO)
            
        \vspace{1.5cm}
            
        \textbf{Łukasz Janusz\\Marek Generowicz}

        \normalsize      
        \textcolor{gray}{09.03.2025}
        \vfill
        \begin{figure}[hb]
            \centering
            \includegraphics[width=0.5\textwidth]{media/Logo_AGH.jpg}
        \end{figure}
                        
            
    \end{center}
\end{titlepage}

\section{Wstęp}
Na laboratoriach należało zapoznać się z protokołem HART, który jest standardem komunikacyjnym stosowanym w przemyśle, zasadami komunikacji oraz praktycznymi aspektami jego wykorzystania w przemyśle. W trakcie zajęć przeprowadzono ćwiczenia z wykorzystaniem sterownika \textit{WAGO 750-841} wyposażonym w dwukanałowy analogowy moduł wejścia, który pozwala na komunikację z urządzeniami HART. Elementem pomiarowym natomiast jest \textit{termopara typu K}, która została połączona z modułem WAGO za pomocą przetwornika temperatury \textit{TxIsoRail-HART}. 

\section{Przebieg ćwiczenia}

\subsection{Konfiguracja PLC}
W pierwszej części zadania należało zaprogramować sterownik \textit{WAGO}. W tym celu należało skorzystać z aplikacji \textit{CoDeSys}. Ważne aby w nowo stworzonym projekcie ustawić \textit{Type od POU} na \textit{Program} a język programowania na \textit{FBD} ze względy na konieczność wykorzystania biblioteki do obsługi \textit{POU} napisanej w tym właśnie języku. Następnie należało dodać moduły wejścia i wyjścia w wirtualnym wnętrzu magistrali. Uzupełniona magistrala wyglądała jak na rysunku \ref{fig1}.

\vspace{1em}
\begin{figure}[ht]
    \centering
    \includegraphics[width=1.3\linewidth]{media/6.png}
    \caption{Wnętrze magistrali w aplikacji \textit{CoDeSys}.}
    \label{fig1}
\end{figure}

Przed przystąpieniem do programowania należało skonfigurować parametry komunikacji oraz, w razie gdyby jej nie było, dodać bibliotekę do obsługi komunikacji HART \textit{WagoLibHART\_03.lib}
\newpage


\subsubsection{Program - HART\_INFOT} \label{1a}
Po połączeniu układu zgodnie z rysunkiem \ref{fig1} należy za pomocą śruby mikrometrycznej znaleźć położenie rdzenia dla, którego napięcie na wyjściu wynosiło 0V, wartość ta wynosi \textit{62,2cm}. 



Następnie należało mierzyć napięcia na wyjściu czujnika w odległość 5 cm od punktu zerowego w obu kierunkach z skokiem o 5 mm. Wartości tych pomiarów przedstawiają trzy pierwsze kolumny tabeli \ref{tab1}.

Na podstawie zmierzonych napięć i przemieszczeń stworzono wykres \ref{wyk1}.



\begin{table}[ht]
    \begin{centering}
    \begin{tabular}{|r|r|r|r|r|}
        \hline
        \makecell{Względne \\przemieszczenie [cm]} & \makecell{Przemieszczenie \\na śrubie [cm]}& \makecell{Napięcie zmierzone [V]}& \makecell{Napięcie\\ obliczone [V]}& \makecell{Moduł \\błędu bezwzględnego}\\\hline
        -5 & 57,2 & 0,59 & -0,586 & 0,004 \\\hline
        -4,5 & 57,7 & 0,53 & -0,5274 & 0,0026 \\\hline
        -4 & 58,2 & 0,47 & -0,4688 & 0,0012 \\\hline
        -3,5 & 58,7 & 0,41 & -0,4102 & 0,0002 \\\hline
        -3 & 59,2 & 0,35 & -0,3516 & 0,0016 \\\hline
        -2,5 & 59,7 & 0,29 & -0,293 & 0,003 \\\hline
        -2 & 60,2 & 0,23 & -0,2344 & 0,0044 \\\hline
        -1,5 & 60,7 & 0,17 & -0,1758 & 0,0058 \\\hline
        -1 & 61,2 & 0,11 & -0,1172 & 0,0072 \\\hline
        -0,5 & 61,7 & 0,053 & -0,0586 & 0,0056 \\\hline
        0 & 62,2 & 0 & 0  & 0 \\\hline
        0,5 & 62,7 & 0,053 & 0,0586 & 0,0056 \\\hline
        1 & 63,2 & 0,11 & 0,1172 & 0,0072 \\\hline
        1,5 & 63,7 & 0,17 & 0,1758 & 0,0058 \\\hline
        2 & 64,2 & 0,23 & 0,2344 & 0,0044 \\\hline
        2,5 & 64,7 & 0,29 & 0,293 & 0,003 \\\hline
        3 & 65,2 & 0,35 & 0,3516 & 0,0016 \\\hline
        3,5 & 65,7 & 0,41 & 0,4102 & 0,0002 \\\hline
        4 & 66,2 & 0,47 & 0,4688 & 0,0012 \\\hline
        4,5 & 66,7 & 0,53 & 0,5274 & 0,0026 \\\hline
        5 & 67,2 & 0,59 & 0,586 & 0,004\\\hline
    \end{tabular}
    \caption{Wyniki pomiarów i obliczeń w zadaniu \textit{1a}.}
    \label{tab1}
    \end{centering}
\end{table}

\vspace{1em}
Kolejno przyjmując ujemne wartości napięcia dla $x < 0$ należy należy aproksymować wartości pomiarów wielomianem stopnia pierwszego, z czego otrzymujemy współczynnik \textit{a = 0,1172} oraz \textit{b = 0}.  Po zastosowaniu wzoru \Ref{eq1} otrzymujemy wartość napięcia obliczonego przedstawione w czwartej kolumnie tabeli \Ref{tab1}. 

\begin{equation} \label{eq1}
    y = ax + b
\end{equation}
\newpage



\begin{graph}[ht]
    \centering
    \includegraphics[width=1\linewidth]{media/Brak_zdjecia.png}
    \caption{Wartość napięcia zmierzonego w zależności od przemieszczenia rdzenia.}
    \label{wyk1}
\end{graph}

Następnie stosując wzór \ref{eq2}, obliczamy wartość błędu bezwzględnego dla każdego pomiaru. Największe błędy bezwzględne wynoszą \textpm 0,0072 i występują w odległości 1 cm od odległości śruby, dla której napięcie zmierzone wynosi 0V. Przebieg wartości błędu bezwzględnego w stosunku do względnego wartości przemieszczenia śruby od stanu początkowego przedstawia wykres \Ref{wyk2}, natomiast moduł wartości błędy bezwzględnego przedstawia ostatnia kolumna tabeli \ref{tab1}.


\begin{equation} \label{eq2}
    \Delta_U = U_{pomiar} - U_{obliczone}
\end{equation}


\newpage
Na koniec zadania \textit{1a} należało obliczyć błąd nieliniowości zgodnie z wzorem \ref{eq3}. Z tabeli możemy odczytać wartość maksymalnego błędu bezwzględnego oraz zakres napięcia wyjściowego czujnika . Po podstawieniu wartości do wzoru otrzymujemy wartość błędu nieliniowości \textit{0,0122}.


\begin{equation} \label{eq3}
    \delta_{U_{max}} = \frac{|\Delta_U|_{max}}{U_{max} - U_{min}}
\end{equation}

\begin{graph}[ht]
    \centering
    \includegraphics[width=1\linewidth]{media/Brak_zdjecia.png}
    \caption{Wartość błędu bezwzględnego w zależności od przemieszczenia rdzenia.}
    \label{wyk2}
\end{graph}

\newpage
\subsubsection{Obserwacja sygnałów na poszczególnych etapach przetwarzania czujnik LVDT – wzmacniacz z modulacją AM}

W celu obserwacji sygnałów na poszczególnych etapach należało układ z rysunku \Ref{fig1} podpiąć do oscyloskopu, który był połączony z komputerem z oprogramowaniem pozwalającym na odczytywanie sygnałów. 
A następnie przeprowadzić pomiary dla rdzenia w pozycji zerowej oraz w pozycji przesuniętej w obu kierunkach.\\
Z powodu braku czasu na wykonanie ostatniego, 2 zadania, w trakcie pracy na oscyloskopie mogliśmy zaobserwować dużo ciekawych zależności i lepiej zrozumieć działanie układu.

\begin{figure}[!ht]
    \centering
    % Pod figura 1
    \subfloat[Pierwszy wykres z oscyloskopu]{
        \includegraphics[width=0.5\textwidth]{media/Brak_zdjecia.png}
        \label{fig:sub1}
    }
    % Pod figura 2
    \subfloat[Drugi wykres z oscyloskopu]{
        \includegraphics[width=0.5\textwidth]{media/Brak_zdjecia.png}
        \label{fig:sub2}
    }
    \hfill
            \subfloat[Trzeci wykres z oscyloskopu]{
                \includegraphics[width=0.5\textwidth]{media/Brak_zdjecia.png}
                \label{fig:sub0}
    }
    
    \caption{Wykresy z oscyloskopu.}
    \label{fig:main}
\end{figure}

\newpage
Na rysunkach \ref{fig:main} i \ref{fig:main1} przedstawione jest pięć różnych zdjęcia stworzonych przez oscyloskop. Na każdym z nich widać 4 wykresy. Pomarańczowy to jest sygnał wejściowy, niebieska to sygnał wyjściowy układu, fioletowy to moduł oddanego sygnału, a zielona to sygnał fazy. Te cztery wyjścia oscyloskopu pozwalają nam odczytać i zrozumieć jak działa nasz układ.

\vspace{1em}
Zdjęcie \ref{fig:sub0} przedstawia wykresy dla rdzenia w pozycji zerowej. Widać że sygnał modułu jest zerowy a sygnał fazy jest na środku. Z tego powodu sygnał wyjścia również się zeruje.

\vspace{1em}
Na zdjęciu \Ref{fig:sub1} widać że sygnał wyjścia jest niewiele przesunięty względem sygnału wejściowego, natomiast na zdjęciu \Ref{fig:sub2} widać że sygnał wyjścia jest w przeciw fazie do sygnału wejściowego. Ponadto moduły są przeciwne względem siebie a wykres fazy na drugim zdjęciu jest znacznie niżej niż na pierwszym zdjęciu. 
Wynika to z tego że odległość rdzenia od początkowej pozycji w trakcie pomiarów różni się o taką samą wartość od położenia zerowego, jednak w przeciwnym kierunku.

\begin{figure}[!ht]
    \centering
    % Pod figura 3
    \subfloat[Czwarty wykres z oscyloskopu]{
        \includegraphics[width=0.5\textwidth]{media/Brak_zdjecia.png}
        \label{fig:sub3}
    }
    % Pod figura 4
    \subfloat[Piąty wykres z oscyloskopu]{
        \includegraphics[width=0.5\textwidth]{media/Brak_zdjecia.png}
        \label{fig:sub4}
    }

    \caption{Wykresy z oscyloskopu.}
    \label{fig:main1}
\end{figure}


\vspace{1em}
Na wykresach \Ref{fig:sub3} i \Ref{fig:sub4} widać że sygnały wyjść zmieniają się bardzo intensywnie, co jest spowodowane bardzo nagłymi zmianami położenia rdzenia względem początkowej pozycji. Pokaz ten został przeprowadzony aby pokazać jak układ zachowuje się w ekstremalnych warunkach oraz jak zmieniają się sygnały wyjściowe przy dużych zmianach położenia rdzenia.

\vspace{1em}
Praca na oscyloskopie pozwoliła nam zobaczyć jak zachowują się wyjścia układu w zależności od położenia rdzenia oraz jakie są różnice między sygnałami wejściowymi a wyjściowymi. Jest to bardzo użyteczna informacja dla osób pracujących z układami pomiarowymi, ponieważ pozwala na zrozumienie jak działa układ oraz jakie są różnice korelacje między wyjściowymi.
\newpage
\subsection{Wyznaczanie charakterystyki statycznej układu czujnik + tor z modulacją amplitudową}

Celem ćwiczenie było wyznaczenie charakterystyki statycznej układu zbudowanego zgodnie z schematem przedstawionym na rysunku \Ref{fig6}, a następnie przeprowadzenie analizy analogicznej jak w zadaniu \textit{1a}, opisanym w punkcie \ref{1a} z tą różnicą że należało nadać sygnał stały.

\begin{figure}[ht]
    \centering
    \includegraphics[width=1\linewidth]{media/Brak_zdjecia.png}
    \caption{Schemat układu do wyznaczania charakterystyki statycznej układu czujnik + tor z modulacją amplitudową.}
    \label{fig6}
\end{figure}
\newpage
Po połączeniu układu zgodnie z rysunkiem \Ref{fig6} należy za pomocą śruby mikrometrycznej znaleźć położenie rdzenia, dla którego napięcie na wyjściu wynosiło 0V, wartość ta wynosi \textit{62,25cm}.\\Następnie przeprowadzone zostały pomiary napięcia na wyjściu czujnika w odległość 5 cm od punktu zerowego w obu kierunkach z skokiem o 5 mm. Wartości tych pomiarów przedstawiają trzy pierwsze kolumny tabeli \Ref{tab2}.

\begin{table}[ht]
    \begin{tabular}{|r|r|r|r|r|}
    \hline
    \makecell{Względne \\przemieszczenie [cm]} & \makecell{Przemieszczenie \\na śrubie [cm]}& \makecell{Napięcie zmierzone [V]}& \makecell{Napięcie\\ obliczone [V]}& \makecell{Moduł \\błędu bezwzględnego}\\\hline
    -5   & 57,25 & 1,06  & 1,05   & 0,01  \\ \hline
    -4,5 & 57,75 & 0,96  & 0,945  & 0,015 \\ \hline
    -4   & 58,25 & 0,85  & 0,84   & 0,01 \\ \hline
    -3,5 & 58,75 & 0,74  & 0,735  & 0,005 \\ \hline
    -3   & 59,25 & 0,64  & 0,63   & 0,01 \\ \hline
    -2,5 & 59,75 & 0,53  & 0,525  & 0,005 \\ \hline
    -2   & 60,25 & 0,43  & 0,42   & 0,01 \\ \hline
    -1,5 & 60,75 & 0,32  & 0,315  & 0,005 \\ \hline
    -1   & 61,25 & 0,21  & 0,21   & 0 \\ \hline
    -0,5 & 61,75 & 0,1   & 0,105  & 0,005 \\ \hline
    0    & 62,25 & 0     & 0      & 0 \\ \hline
    0,5  & 62,75 & -0,11 & -0,105 & 0,005 \\ \hline
    1    & 63,25 & -0,21 & -0,21  & 0 \\ \hline
    1,5  & 63,75 & -0,32 & -0,315 & 0,005 \\ \hline
    2    & 64,25 & -0,43 & -0,42  & 0,01 \\ \hline
    2,5  & 64,75 & -0,53 & -0,525 & 0,005 \\ \hline
    3    & 65,25 & -0,64 & -0,63  & 0,01 \\ \hline
    3,5  & 65,75 & -0,74 & -0,735 & 0,005 \\ \hline
    4    & 66,25 & -0,85 & -0,84  & 0,01 \\ \hline
    4,5  & 66,75 & -0,96 & -0,945 & 0,015 \\ \hline
    5    & 67,25 & -1,06 & -1,05  & 0,01 \\ \hline
    \end{tabular}
    \caption{Wyniki pomiarów i obliczeń w zadaniu \textit{3}.}
    \label{tab2}
\end{table}
\newpage
\vspace{1em}
Następnie na podstawie zmierzonych napięć i przemieszczeń stworzono wykres \Ref{wyk3}. Z wykresu tego aproksymowane zostały wartości współczynników \textit{a = -0,21} oraz \textit{b = 0}. Po zastosowaniu wzoru \Ref{eq1} otrzymujemy wartość napięcia obliczonego przedstawione w czwartej kolumnie tabeli \Ref{tab2}.


\begin{graph}[ht]
    \centering
    \includegraphics[width=1\linewidth]{media/Brak_zdjecia.png}
    \caption{Wartość napięcia zmierzonego w zależności od przemieszczenia rdzenia.}
    \label{wyk3}
\end{graph}
\newpage
Posiadając wartości napięć zmierzonych i obliczonych, obliczono wartość błędu bezwzględnego dla każdego pomiaru zgodnie z wzorem \Ref{eq2}. Największe błędy bezwzględne wynoszą \textpm 0,015 i występują dla w odległości 4,5 cm od odległości śruby dla, której napięcie zmierzone wynosi 0V. Przebieg wartości błędu bezwzględnego w stosunku do względnego wartości przemieszczenia śruby od stanu początkowego przedstawia wykres \Ref{wyk4}, natomiast moduł wartości błędu bezwzględnego przedstawiają ostatnia kolumna tabeli \Ref{tab2}.

\begin{graph}[ht]
    \centering
    \includegraphics[width=1\linewidth]{media/Brak_zdjecia.png}
    \caption{Wartość błędu bezwzględnego w zależności od przemieszczenia rdzenia.}
    \label{wyk4}
\end{graph}
Na koniec zadania \textit{3} obliczono błąd nieliniowości zgodnie z wzorem \Ref{eq3}. Z tabeli możemy odczytać wartość maksymalnego błędu bezwzględnego oraz zakres napięcia wyjściowego czujnika po podstawieniu wartości do wzoru otrzymujemy wartość błędu nieliniowości \textit{0,007}.
\newpage

\section{Podsumowanie}
Sprawozdanie dotyczy badania właściwości metrologicznych toru pomiarowego z modulacją AM współpracującego z transformatorowym czujnikiem LVDT. Realizowane zadania obejmowały wyznaczanie charakterystyk statycznych czujnika oraz układu czujnik + tor, analizę błędów, a także obserwację sygnałów na różnych etapach przetwarzania.


W ramach badań wyznaczono charakterystykę statyczną czujnika LVDT, mierząc napięcie wyjściowe w funkcji przemieszczenia rdzenia. Uzyskane dane wskazują na dobrą liniowość czujnika w badanym zakresie, z maksymalnym błędem nieliniowości wynoszącym 0,0122. Analiza sygnałów na oscyloskopie pozwoliła zobrazować dynamikę układu czujnik-wzmacniacz oraz jego reakcje na różne pozycje rdzenia i szybkie zmiany położenia. Wyznaczono również charakterystykę statyczną całego układu czujnik + tor, obliczając błędy i uzyskując błąd nieliniowości na poziomie 0,007, co wskazuje na wysoką zgodność z modelem teoretycznym.
\begin{enumerate}
    \item \textbf{Właściwości czujnika LVDT:} Czujnik wykazuje dobrą liniowość i precyzję w badanym zakresie, co potwierdza jego przydatność w precyzyjnych pomiarach przemieszczeń.
    \item \textbf{Skuteczność modulacji AM:} Wzmacniacz z modulacją AM skutecznie przetwarza sygnały z czujnika, minimalizując zakłócenia.
    \item \textbf{Praktyczne zastosowania:} Układ może być stosowany w systemach pomiarowych wymagających wysokiej dokładności, takich jak mechanika precyzyjna.
    \item \textbf{Obserwacje oscyloskopowe:} Analiza sygnałów umożliwiła lepsze zrozumienie dynamiki układu oraz korelacji między sygnałami wejściowymi a wyjściowymi.
\end{enumerate}
Podsumowując, przeprowadzone badania wykazały, że tor pomiarowy z czujnikiem LVDT i wzmacniaczem z modulacją AM spełnia założenia metrologiczne, oferując wysoką dokładność i stabilność pomiarów.

\end{document}
